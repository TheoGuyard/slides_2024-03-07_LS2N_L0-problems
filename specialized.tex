\section{Specialized Solution Methods}

\begin{frame}{Branch-and-Bound algorithms}
    \begin{tikzpicture}[remember picture,overlay]
        \node[align=center,text width=0.9\textwidth] (concept) at ($(current page.north)+(0,-0.2\textwidth)$) {\textbf{Branch-and-Bound} \\ \textit{``Enumerate all candidate solutions and discard sub-optimal ones.''}};
        %
        %
        %
        \node (path0) at ($(current page.north)+(1.5,-3.5)$) {};
        \node (path1) at ($(path0)+(2,0)$) {};
        \node (path2) at ($(path0)+(3,-1)$) {};
        \node (path3) at ($(path0)+(2,-3)$) {};
        \node (path4) at ($(path0)+(1,-1.5)$) {};
        \node (path5) at ($(path0)+(0,-1)$) {};
        \path[ultra thick,draw,use Hobby shortcut,closed=true] (path0)..(path1)..(path2)..(path3)..(path4)..(path5);
        %
        %
        %
        \node at ($(path0)+(2,-0.75)$) {\Large{$\bullet$}};
        \node at ($(path0)+(1,-0.75)$) {\Large{$\bullet$}};
        \node at ($(path0)+(2.75,-2)$) {\Large{$\bullet$}};
        \node at ($(path0)+(1.75,-2)$) {\Large{$\bullet$}};
        %
        %
        %
        \draw[ultra thick,dashed] (path2) -- (path4);
        %
        %
        %
        \draw[ultra thick,dashed] ($(path4)+(0.75,0.2)$)-- ($(path1)+(-1,0.4)$);
        \draw[ultra thick,dashed] ($(path4)+(1,0.2)$) -- ($(path3)+(0.5,0)$);
        %
        %
        %
        \node (N3cross) at ($(path0)+(1,-0.75)$) {\textcolor{purple}{\LARGE\ding{55}}}; 
        \node (N5cross) at ($(path0)+(1.75,-2)$) {\textcolor{purple}{\LARGE\ding{55}}}; 
        \node (N6cross) at ($(path0)+(2.75,-2)$) {\textcolor{purple}{\LARGE\ding{55}}}; 
        %
        %
        %
        \node[text width=\textwidth] (principles1) at ($(current page.north)+(2,-0.65\textwidth)$) {
            \textbf{\hspace*{-1.5cm}Main principles}
            \begin{itemize}
                \item[Branching:] Divide the search space
                \item[Bounding:] Test whether a region can contain optimal solutions 
                \item[Pruning:] Discard regions without optimal solutions
            \end{itemize}
        };
    \end{tikzpicture}
\end{frame}

\begin{frame}{Tree exploration}
    \begin{tikzpicture}[remember picture,overlay]
        \onslide<+-> {
            \node at ($(current page.north)+(0,-0.45\textheight)$) (node0) {};
            \draw[
                ultra thick,
                top color = white,
                bottom color = blue!30,
            ] (node0) circle (15pt) node {\Large$N_0$};
            %
            \node at ($(node0)+(-3,-1.5)$) (node1) {};
            \draw[
                ultra thick,
                top color = white,
                bottom color = blue!30,
            ] (node1) circle (15pt) node {\Large$N_1$};
            \draw[ultra thick,->] ($(node0.south west)+(-0.25,-0.25)$) -- ($(node1.north east)+(0.25,0.25)$) node[midway,fill=white,draw,font=\scriptsize] {$\pvi{1} = 0$};
            %
            \node at ($(node0)+(3,-1.5)$) (node2) {};
            \draw[
                ultra thick,
                top color = white,
                bottom color = blue!30,
            ] (node2) circle (15pt) node {\Large$N_2$};
            \draw[ultra thick,->] ($(node0.south east)+(0.25,-0.25)$) -- ($(node2.north west)+(-0.25,0.25)$) node[midway,fill=white,draw,font=\scriptsize] {$\pvi{1} \neq 0$};
            %
            \node at ($(node1)+(-1,-2)$) (node3) {};
            \draw[
                ultra thick,
                top color = white,
                bottom color = blue!30,
            ] (node3) circle (15pt) node {\Large$N_3$};
            \draw[ultra thick,->] ($(node1.south west)+(-0.25,-0.25)$) -- ($(node3.north)+(0.25,0.35)$) node[midway,fill=white,draw,font=\scriptsize] {$\pvi{2} = 0$};
            %
            \node at ($(node1)+(1,-2)$) (node4) {};
            \draw[
                ultra thick,
                top color = white,
                bottom color = blue!30,
            ] (node4) circle (15pt) node {\Large$N_4$};
            \draw[ultra thick,->] ($(node1.south east)+(0.25,-0.25)$) -- ($(node4.north)+(-0.25,0.35)$) node[midway,fill=white,draw,font=\scriptsize] {$\pvi{2} \neq 0$};
            %
            %
            %
            \node at ($(node2)+(-1,-2)$) (node5) {};
            \draw[
                ultra thick,
                top color = white,
                bottom color = blue!30,
            ] (node5) circle (15pt) node {\Large$N_5$};
            \draw[ultra thick,->] ($(node2.south west)+(-0.25,-0.25)$) -- ($(node5.north)+(0.25,0.35)$) node[midway,fill=white,draw,font=\scriptsize] {$\pvi{2} = 0$};
            \node at ($(node2)+(1,-2)$) (node6) {};
            \draw[
                ultra thick,
                top color = white,
                bottom color = blue!30,
            ] (node6) circle (15pt) node {\Large$N_6$};
            \draw[ultra thick,->] ($(node2.south east)+(0.25,-0.25)$) -- ($(node6.north)+(-0.25,0.35)$) node[midway,fill=white,draw,font=\scriptsize] {$\pvi{2} \neq 0$};
        }
        %
        %
        %
        \onslide<+-> {
            \node[align=center,text width=0.41\textwidth,font=\small] (observation0) at ($(current page.north)+(-3.7,-2.2)$) {
                \begin{blockcolor}{black}{Observation}
                    If \emphcolor{Brown}{support of $\pv$ is fixed}, then \\
                    $\quad\min_{\pv} \lossfunc(\pv) + \reg\norm{\pv}{0} + \pertfunc(\pv)$ \\
                    is easy to solve.
                \end{blockcolor}
            };
        }
        %
        %
        %
        \onslide<+-> {
            \node[font=\scriptsize,align=left] at ($(node0)+(1,0)$) (pb0) {\emphcolor{Green}{$\texttt{lb} = 1$} \\ \emphcolor{Brown}{$\texttt{ub} = 5$}};
            %
            \node[font=\scriptsize,align=left] at ($(node1)+(-1.1,0)$) (pb1) {\emphcolor{Green}{$\texttt{lb} = 1.5$} \\ \emphcolor{Brown}{$\texttt{ub} = 4.5$}};
            %
            \node[font=\scriptsize,align=left] at ($(node3)+(-1.1,0)$) (pb3) {\emphcolor{Green}{$\texttt{lb} = 2$} \\ \emphcolor{Brown}{$\texttt{ub} = 4.5$}};
            %
            \node[font=\scriptsize,align=left] at ($(node4)+(1.1,0)$) (pb4) {\emphcolor{Green}{$\texttt{lb} = 2.5$} \\ \emphcolor{Brown}{$\texttt{ub} = 3$}};
            %
            \node[font=\scriptsize,align=left] at ($(node2)+(1.1,0)$) (pb2) {\emphcolor{Green}{$\texttt{lb} = 2$} \\ \emphcolor{Brown}{$\texttt{ub} = 4.5$}};
            %
            \node[font=\scriptsize,align=left] at ($(node5)+(-1.1,0)$) (pb5) {\emphcolor{Green}{$\texttt{lb} = 3.5$} \\ \emphcolor{Brown}{$\texttt{ub} = 4$}};
            %
            \node[font=\scriptsize,align=left] at ($(node6)+(1.1,0)$) (pb6) {\emphcolor{Green}{$\texttt{lb} = 4$} \\ \emphcolor{Brown}{$\texttt{ub} = 4.5$}};
        }
        %
        %
        %
        \onslide<+-> {
            \draw[ultra thick,-{Circle[scale=0.5]}] ($(node5.south)+(0,-0.4)$) -- ($(node5.south)+(0,-0.6)$);
        }
        %
        %
        %
        \onslide<+-> {
            \draw[ultra thick,->] ($(node4.south east)+(0.25,-0.25)$) -- ($(node4.south east)+(0.5,-0.5)$);
            \draw[ultra thick,->] ($(node4.south west)+(-0.25,-0.25)$) -- ($(node4.south west)+(-0.5,-0.5)$);
            \draw[ultra thick,->] ($(node3.south east)+(0.25,-0.25)$) -- ($(node3.south east)+(0.5,-0.5)$);
            \draw[ultra thick,->] ($(node3.south west)+(-0.25,-0.25)$) -- ($(node3.south west)+(-0.5,-0.5)$);
            \draw[ultra thick,-{Circle[scale=0.5]}] ($(node6.south)+(0,-0.4)$) -- ($(node6.south)+(0,-0.6)$);
        }
        %
        %
        %
        \onslide<+-> {
            \node[align=center] (convcrit) at ($(current page.north)+(-2,-1.05\textheight)$) {\small{All nodes explored or pruned}};
            \node[align=center] (convres) at ($(convcrit)+(4,0)$) {\small{Problem solved}};
            \draw[ultra thick,->] (convcrit.east) -- (convres.west);
        }
    \end{tikzpicture}
\end{frame}

\begin{frame}{Node processing}
    \begin{tikzpicture}[remember picture,overlay]
        \onslide<+-> {
            \node at ($(current page.north)+(-5,-0.325\textheight)$) (node) {};
            \draw[
                ultra thick,
                top color = white,
                bottom color = blue!30,
            ] (node) circle (15pt) node {\Large$\nodeSymbIter{}_k$};
            \draw[ultra thick,->] ($(node.north)+(0,0.75)$) -- ($(node.north)+(0,0.4)$);
            \node at ($(node.south)+(0,-0.75)$) (nodesets) {$(\setzero,\setone)$};
            %
            \node[align=center,text width=0.7\textwidth,font=\small] (nodeproblem) at ($(current page.north)+(0,-0.3\textheight)$) {
                \begin{blockcolor}{black}{Node problem}
                    The problem at node \emphcolor{Brown}{$\nodeSymb = (\setzero,\setone)$} where $\setzero$ and $\setone$ are the indices of $\pv$ fixed to \emphcolor{Brown}{zero} and \emphcolor{Brown}{non-zero} reads
                    \begin{center}
                        $\node{\pobj} = \begin{cases}
                            \min_{\pv} &\lossfunc(\pv) + \reg\norm{\pv}{0} + \pertfunc(\pv) \\
                            \text{s.t.} &\emphcolor{Brown}{\pv_{\setzero} = \0, \ \pv_{\setone} \neq \0} 
                        \end{cases}$
                    \end{center}
                \end{blockcolor}
            };
        }
        %
        %
        %
        \onslide<+-> {
            \node[align=center,text width=0.75\linewidth] (ub) at ($(nodeproblem.south)+(0,-0.05\textwidth)$) {\textbf{Task:} Find lower and upper bounds on $\node{\pobj}$ that are \textbf{tight} and \textbf{tractable to compute}};
        }
        %
        %
        %
        \onslide<+-> {
            \node[align=left,text width=\linewidth] (ub) at ($(current page.north)+(0,-0.625\textwidth)$) {\textbf{Upper bounding} \\ $\qquad\bullet$ We just need a \emphcolor{Brown}{feasible} solution \\ $\qquad\bullet$ Fix entries of $\pv$ that are still free to zero \\ $\qquad\bullet$ Optimize the resulting problem};
        }
        %
        %
        %
        \onslide<+-> {
            \node[align=center,text width=0.5\textwidth,font=\small] (ubproblem) at ($(current page.north)+(0,-0.975\textheight)$) {
                \begin{blockcolor}{Brown}{Upper-bounding problem}
                    \begin{center}
                        $\min_{\pv} \ \lossfunc(\pv_{\setone}) + \reg\card{\setone} + \pertfunc(\pv_{\setone})$
                    \end{center}
                \end{blockcolor}
            };
            %
            \draw[ultra thick,<-,Brown] ($(ubproblem.east)+(0.1,-0.1)$) .. controls ($(ubproblem.east)+(0.75,0.1)$) .. ($(ubproblem.east)+(1,0.5)$) node (easy) [font=\small,above] {Easy to solve};
            \node (nerd) at ($(easy.east)+(0.3,0)$) {\includegraphics[width=0.05\textwidth]{imgs/nerd_face.png}};
        }
    \end{tikzpicture}
\end{frame}

\begin{frame}{Lower bounding}
    \begin{tikzpicture}[remember picture,overlay]
        \onslide<+-> {
            \node[anchor=west] (objective) at ($(current page.north)+(-5.5,-1.5)$) {\textbf{Idea:} 
            Convexify a \emphcolor{Brown}{part} of the objective function};
            %
            %
            %
            \node[text width=.55\textwidth] at ($(current page.north)+(0,-0.35\textheight)$) (pertnodeproblem) {
                \begin{blockcolor}{black}{Node problem}
                    \centering
                    \small
                    $\node{\pobj} = 
                        \left\{
                        \begin{array}{rl}
                            \min_{\pv} & \lossfunc(\pv) + \emphcolor{Brown}{\reg \norm{\pv}{0} + \pertfunc(\pv)} \\
                            \text{s.t.} & \emphcolor{Brown}{\subzero{\pv} = \0, \ \subone{\pv} \neq \0}
                        \end{array}
                        \right.
                    $
                \end{blockcolor}
            };
            %
            \draw[ultra thick,<-] ($(pertnodeproblem.east)+(0.1,-0.1)$) .. controls ($(pertnodeproblem.east)+(0.75,-0.1)$) .. ($(pertnodeproblem.east)+(1.5,-0.5)$) node[font=\small,below] {$\min_{\pv} \lossfunc(\pv) + \emphcolor{Brown}{\regfunc(\pv)}$};
        }
        %
        %
        %
        \onslide<+-> {
            \node[text width=.44\textwidth] at ($(pertnodeproblem.south)+(0,-0.3\textwidth)$) (pertlbproblem) {
                \begin{blockcolor}{Brown}{Lower-bounding problem}
                    \centering
                    $\min_{\pv} \lossfunc(\pv) + \biconj{\regfunc}(\pv)$
                \end{blockcolor}
            };
            %
            \draw[ultra thick,->] ($(pertnodeproblem.south)+(0,0)$) -- ($(pertlbproblem.north)+(0,-0.3)$) node[midway,draw,fill=white,font=\small] (pertnodeproblemtext) {Bi-conjugacy $\biconj{\regfunc}(\pv) \leq \regfunc(\pv)$};
        }
        %
        %
        %
        \onslide<+-> {
            \node[font=\small] (solution) at ($(pertnodeproblemtext)+(4.5,-0.075\textwidth)$) {\textbf{Closed-form expression}};
            \node (happy) at ($(solution)+(0,-0.75)$) {
                \includegraphics[width=20pt]{imgs/party.png}
            };
            %
            \node[align=center, text width=0.3\textwidth] at ($(happy.south)+(0,-0.5)$) {\scriptsize{$\pertfunc$ proper, closed, coercive}};
            \node[align=center, text width=0.3\textwidth] at ($(happy.south)+(0,-0.75)$) {\scriptsize{continuous at its minimum}};
        }
    \end{tikzpicture}
\end{frame}

\begin{frame}{Graphical intuition}
    \newcommand{\pointsize}{0.03}
    \begin{tikzpicture}[
        remember picture,
        overlay,
        domain=-0.8:0.8,
    ]
        \begin{scope}[shift={(2,0)},xscale=2.3,yscale=2.5]
            \node (origin) at (0,0) {};
            \draw[ultra thick,->] (-1,0) -- (1, 0);
            \draw[ultra thick,->] (0,0) -- (0, 1);
            \draw[ultra thick] (-0.03,0.2) -- (0.03,0.2);
            \node[above right] at (0,0.2) {$\reg$};
            \node[right] at (1,0) {$\pv$};
            \node[above] at (0,1) {\small{$\pertfunc \equiv$ Big-M}};
            %
            \draw[{Arc Barb[arc=130,reversed]}-,Teal,ultra thick] (-0.5,0.2) plot[domain=0.03:0.5] (\x,0.2);
            \draw[-{Arc Barb[arc=130,reversed]},Teal,ultra thick] (-0.5,0.2) plot[domain=-0.5:-0.03] (\x,0.2) node {};
            \fill[Teal] (0,0) circle (\pointsize);
            \draw[Teal,ultra thick] (0.5,0.2) -- (0.5,0.85);
            \draw[Teal,thick] (-0.5,0.2) -- (-0.5,0.85);
            %
            %
            %
            \draw[ultra thick,densely dashed] (0, 0) -- (1, 0.4);
            %
            %
            %
            \draw[ultra thick] (0.5,-0.03) -- (0.5,0.03);
            \node[below] at (0.5,0) {\small{$\pertlimit$}};
            \draw[ultra thick,densely dashed] (0.5,0) -- (0.5,0.2);
            \node[rotate=25,anchor=center] at (0.8,0.25) {\scriptsize{slope $\pertslope$}};
            %
            %
            %
            \draw[Brown,ultra thick] (0, 0) -- (0.5, 0.2);
            \fill[Teal] (0,0) circle (\pointsize);
            %
            %
            %
            \draw[Brown,ultra thick] (0.51,0.2) -- (0.51,0.85);
            %
            %
            %
            \draw[Brown,ultra thick] (0, 0) -- (-0.5, 0.2);
            \draw[Brown,ultra thick] (-0.51,0.2) -- (-0.51,0.85);
            \draw[ultra thick] (-0.5,-0.03) -- (-0.5,0.03);
            \node[below] at (-0.5,0) {\small{$-\pertlimit$}};
            \fill[Teal] (0,0) circle (\pointsize);
            %
            %
            %
            \node at (0.6,0.95) {\small\emphcolor{teal}{$\regfunc$}/\emphcolor{Brown}{$\biconj{\regfunc}$}};
        \end{scope}
        %
        %
        %
        \begin{scope}[shift={(8,0)},xscale=2.3,yscale=2.5]
            \node (origin) at (0,0) {};
            \draw[ultra thick,->] (-1,0) -- (1, 0);
            \draw[ultra thick,->] (0,0) -- (0, 1);
            \draw[ultra thick] (-0.03,0.2) -- (0.03,0.2);
            \node[above right] at (0,0.2) {$\reg$};
            \node[right] at (1,0) {$\pv$};
            \node[above] at (0,1) {\small{$\pertfunc \equiv$ L2-norm}};
            \draw[{Arc Barb[arc=130,reversed]}-,Teal,ultra thick] (-0.5,0.2) plot[domain=0.03:0.8] (\x,0.2+\x^2);
            \draw[-{Arc Barb[arc=130,reversed]},Teal,ultra thick] (-0.5,0.2) plot[domain=-0.8:-0.03] (\x,0.2-\x^2) node {};
            \fill[Teal] (0,0) circle (\pointsize);
            %
            %
            %
            \draw[ultra thick,densely dashed] (0, 0) -- (1, 0.86);
            %
            %
            %
            \node[below] at (0.5,0) {\small{$\pertlimit$}};
            \draw[ultra thick,densely dashed] (0.5,0) -- (0.5,0.42);
            %
            %
            %
            \node[rotate=45,anchor=center] at (0.28,0.17) {\scriptsize{slope $\pertslope$}};
            %
            %
            %
            \draw[Brown,ultra thick] (0, 0) -- (0.5, 0.43);
            %
            %
            %
            \draw[Brown,ultra thick] (0.5, 0.43) plot[domain=0.5:0.8] (\x,0.2+\x^2-0.02) node {};
            %
            %
            %
            \draw[Brown,ultra thick] (0, 0) -- (-0.5, 0.43);
            \draw[Brown,ultra thick] (-0.5, 0.43) plot[domain=-0.8:-0.5] (\x,0.2-\x^2-0.02) node {};
            \fill[Teal] (0,0) circle (\pointsize);
            %
            \draw[ultra thick] (-0.5,-0.03) -- (-0.5,0.03);
            \node[below] at (-0.5,0) {\small{$-\pertlimit$}};
            \draw[ultra thick] (0.5,-0.03) -- (0.5,0.03);
            %
            %
            %
            \node at (0.8,0.95) {\small\emphcolor{teal}{$\regfunc$}/\emphcolor{Brown}{$\biconj{\regfunc}$}};
        \end{scope}
        %
        %
        %
        \node[text width=.5\textwidth] at ($(current page.north)+(0,-0.675\textwidth)$) (closedform) {
            \begin{blockcolor}{black}{Bi-conjugate closed-form}
                \small
                \centering
                $\biconj{\regfunc}(\pv) = 
                \begin{cases}
                    \emphcolor{Brown}{\pertslope}\abs{\pv} & \text{if} \ \abs{\pv} \leq \emphcolor{Brown}{\pertlimit} \\
                    \regfunc(\pv) & \text{otherwise}
                \end{cases}$
            \end{blockcolor}
        };
    \end{tikzpicture}
\end{frame}

\begin{frame}{Let's sum up !}
    \begin{tikzpicture}[remember picture,overlay]
        \node [text width=.4\textwidth] at ($(current page.north)+(0,-0.25\textheight)$) (problem) {
            \begin{blockcolor}{Brown}{$\ell_0$-penalized problem}
                \centering
                $\min_{\pv} \ \lossfunc(\pv) + \reg \norm{\pv}{0} + \pertfunc(\pv)$
            \end{blockcolor}
        };
        \node [text width=1.1\textwidth] at ($(current page.north)+(0,-0.7\textheight)$) (observations) {
            \begin{itemize}[label=$\blacktriangleright$]
                \item MIP formalism
                \begin{itemize}[label=$\bullet$]
                    \item Linearize the $\ell_0$-norm with a binary variable
                    \item Big-M strategy
                \end{itemize}
                \item Generic solvers
                \begin{itemize}[label=$\bullet$]
                    \item Easy solution to implement
                    \item Unable to exploit sparsity
                    \item Numerically inefficient
                \end{itemize}
                \item Specialized Branch-and-Bound 
                \begin{itemize}[label=$\bullet$]
                    \item Tree exploration
                    \item Branch by fixing support of $\pv$
                    \item Compute upper and lower bounds at each node
                    \item Leverage bi-conjugacy to compute lower bounds
                \end{itemize}
            \end{itemize}
        };
    \end{tikzpicture}
\end{frame}